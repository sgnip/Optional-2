\documentclass{article}

\usepackage{fancyhdr} % Cabeceras de página
\usepackage{lastpage} % Módulo para añadir una referencia a la última página
\usepackage{titling} % No tengo claro para qué es esto
\usepackage[left=2cm,right=2cm,top=3cm,bottom=2cm]{geometry} % Márgenes
\usepackage{xspace}
\usepackage{graphicx}
\usepackage{tikz}
\usepackage{float}

\title{Optional Assignment 2}
\date{\today}
\author{SGNIP company}

\fancyhf{}
\fancypagestyle{plain}{%
	\lhead{\small \itshape \thetitle\, -\, \thedate\, -\, SEPRO}
	\rhead{\vspace{-20pt} \includegraphics[width=40 pt]{Logo.jpg}}
	\cfoot{\thepage\ of \pageref{LastPage}}
	\rfoot{}
}

\begin{document}

\maketitle

\newpage

\section{Problem 1}

\subsection{Random brainstorming to answer}

\begin{itemize}
	\item Meet and get to know as much as possible people I'm gonna work with, so we can work together being more productive.
	\subitem Assign roles to people in the team (if needed).
	\item Monitor:
	\subitem	Expenses.
	\subitem Timing plan.
	\subitem Performed effort.
	\subitem Consumed resources.
	\subitem Technical	activities.
	\subitem Results.
	\subitem Risks.
	\subitem Unsolved conflicts or problematic aspects.
	\item Selection of standards, methods, procedures
	\item Risks identification and contingency plans.
	\item Identification of reusable ‘products’/similar projects.
	
\end{itemize}

\subsection{Priorized task}


\begin{enumerate}
\item Identification of reusable ‘products’/similar projects.
\item Selection of methodology to work with. It may be interesting to take into account possible feedback to TAKEOFF Ltd.
\item Selection of standards, methods, procedures, so everybody can start working on the project. If there is some similar project, we should choose technology depending on that.

If it's not done yet allocating responsibilities and activities.
(We guess it's, because before analyzing requirements, the person in charge of analysis must be already allocated with it's responsibility and it's activity)\item Meet and get to know as much as possible people I'm gonna work with, so we can work together being more productive.
\subitem 3.1 Assign roles to people in the team (if it's not done yet).
\item Risks identification and contingency plans. Now developers and programmers are working, we can define contingency plans for risks after we identify them.
\item Monitor software quality, technical results to assure the software won't fail.
\item Monitor timing plan trying to assure it's not delayed. This is not so important as quality because we have 4 months of acceptable delay.
\item Product and process review and give feedback to TAKEOFF Ltd. if needed.

\end{enumerate}


\section{Problem 2}

\paragraph{Applying a variant of the field observation technique to the set film, list the reasons why your business might fail. What measures would you take to prevent each of the mistakes from causing your business to fail?}



\begin{itemize}
	\item Lack of business strategy and being too optimistic.
	\item Tension inside the team.
	\item Lack of senior executives increased because of no previous failures.
	\item No competitors analysis.
	\item No ability to keep up with technology to take care of mass problem (website).
	\item Too rapid grow with no experience neither exhaustive plan to control it.

\end{itemize}

\paragraph{Shifting the focus of the film to software project definition and start-up, what problems might arise and how could they be solved if the customer, developer and user are all the same person and if they are different people? What are the potential advantages of the customer, developer and user being the same person?}


The biggest advantage given by  customer and developer being the same person is the impossibility of lack of communication. One should know exactly what he wants, so the requirements are clear from the very beginning.

The same happens if the developer and the user are the same person. 

Even easier communication would be if customer and user are also the same person. What the user needs is exactly what the customer would want to to be developed.

In summary, the so normal communication problem would not exist alt all.

\paragraph{What qualities should or must a project leader have?}
The first of all, being communicative. Being able to demand the amount of work necessary to the workers, without make them slaves. A project leader should know the workers he works with to be able to demand work properly.

A project leader must know some technical stuff involved in the project for a better comprehension of it.

Another important quality is knowing how to solve problems among the workers.


\paragraph{What project management and leadership activities are performed in Startup.com?}

In Startup.com is performed a greedy aim to get money and get clients. Is also performed a poor execution of plans. As a leaders, is performed a poor example to the workers because of the tension between the founders.

\paragraph{List the steps to be taken to define and start up a software project}

\begin{enumerate}
	\item Have a great idea.
	\item Check it is possible and nobody else has done it.
	\subitem if that idea exists, think how to improve it.
	\item Get a small great team.
	\item Viability study to see if it is possible to start up.
	\item Get an office and some investors who can help you with first period of businesses where no money is earned.
	\item Start working, in developing (not just coding, defining requirements is developing too) and in commercial labors: someone will need to buy what you are doing so the businesses makes money.
	\item Know how to deal with rapid grow. Know how to let others do the job so you don't have to be in charge of everything.
	\item Now you have made a great businesses, you can keep working and earning money.
\end{enumerate}

\end{document}
