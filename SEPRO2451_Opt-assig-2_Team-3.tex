\documentclass{article}

\usepackage{fancyhdr} % Cabeceras de página
\usepackage{lastpage} % Módulo para añadir una referencia a la última página
\usepackage{titling} % No tengo claro para qué es esto
\usepackage[left=2cm,right=2cm,top=3cm,bottom=2cm]{geometry} % Márgenes
\usepackage[T1]{fontenc}
\usepackage[utf8x]{inputenc}
\usepackage{xspace}
\usepackage{graphicx}
\usepackage{tikz}
\usepackage{wrapfig}
\usepackage{hyperref}
\usepackage{amssymb}

\usepackage{float}

\title{Optional Assignment 2}
\date{\today}
\author{{\Large Triforce} \\ \vspace{5pt} \textit{Iván Márquez Pardo, Víctor de Juan Sanz, Guillermo Julián Moreno}}

\fancyhf{}
\fancypagestyle{plain}{%
	\lhead{\small \itshape \thetitle\, -\, \thedate\, -\, SEPRO}
	\rhead{\vspace{-20pt} \includegraphics[width=40 pt]{Logo.jpg}}
	\cfoot{\thepage\ of \pageref{LastPage}}
	\rfoot{}
}

\begin{document}

\maketitle

\newpage

\section{Problem 1}
\paragraph{As project leader, what would your activities during the development phase be?
Prioritize these activities.}
\paragraph{}

We have defined the important tasks to be done in the order we think is the proper one.

\subsection{Prioritized task}
\begin{enumerate}

\item First of all, since I am the new project leader of a project I am not familiarized with, I have to get an idea of the current status of it. I must analyze the problem in detailed and see how the analysis phase has been developed. Design development phase will be based on this analysis so I have to get in contact with it if I want to know how to organize future tasks. 

\item Given the objective of the project, it is feasible that similar projects have already been developed. Those projects would be an extremely useful base point from where I could organize mine.

\item Once I completely comprehend the scope and objectives of the problem, it is time to know the team I am going to work with. A first meeting with the whole team is needed in order to know how things have been done until my arrival. If I find the previous method of work appropiate and the development team was comfortable with it, I would maintain it and try to adapt it; if not, then I would state my own method in order to achieve the goals we have proposed.

\item The analysis phase carried out by the previous project leader must have generated a Software Requirements Specification. After looking carefully at it, if I find it inaccurate or ambiguous, getting back to that phase would be needed to complete that document. I hope it won't be neccessary if it was correctly develop back then.

\item Finished the analysis, we have to make the selection of the methodology we plan to work with for the rest of the project. 

\item Revise the cycle of life of this software product. A satellite is a huge investment, so its orbit control software has to be refined to the max so that we don't waste all that money. Taking into account the huge quantity of risks of this project, I consider beforehand that an spiral cycle of life would be appropiate for it and would give us the reliability we need.

\item TAKEOFF Ltd. will surely appreciate some periodical feedback explaining the current status of the software project and the problems (if any) we are dealing with. One report alternate weeks should be enough. Product and process reviews will also be generated during the development.

\item Selection of standards, methods, procedures, and hardware and software environment, so everybody can start working on the project. If there is any similar project, we should choose technology depending on that.

\item During the design phase, we have to take into account that the data acquisition module has been requested to be very generic, so we will have to do an special effort to make its design abstract and generally applicable.

\item Risks identification and contingency plans. Now that developers and programmers are working, we can define contingency plans for risks just after we identify them.

\item Both design and codification will be performed taking into account the minimum maintenance we will be able to give it once the satellite has launched. 

\item After the codification phase, a huge quantity of time must be reserved for the test phase. This phase is critical because of the nature of the project. Moreover, as we have been requested to give our software a huge grade of correctness, the test phase must be carefully performed, even if that means delaying the project development by a whole month.

\item Monitor software quality, technical results to assure the software won't fail.

\item Monitor timing plan trying to assure it's not delayed. This is not so important as quality because we have 4 months of acceptable delay.

\item While the development team is working on the project the first two weeks, if we have enough time available, I would like to delegate responsabilities in the most experienced member of the development team for 1-3 days. During that time, I would be able to meet personally the rest of the members of the development team and make myself an idea of their abilities in order to improve future task assignments and responsabilities and activities allocation. This way, I may increase both coordination and productivity.

\item As I find important the motivation of my team, I plan to fix the Informal-Friday in the office and bring some doughnuts on Mondays in order to improve the comfortability and the working environment of my employees. Moreover, I would put a coffee machine in the office, sattisfying my team and improving their performance on the job (because they won't fall asleep early in the morning).

\item If we finally finish the project at the time or we don't surpass the 2-3 months of delay (the time limit is in 4 months, but in that case, there won't be any reward), I will organize a company dinner in order to celebrate this success.

\end{enumerate}

\newpage
\section{Problem 2}

\paragraph{Applying a variant of the field observation technique to the set film, list the reasons why your business might fail. What measures would you take to prevent each of the mistakes from causing your business to fail?}
\paragraph{}

This is the list of possible mistakes that can cause some business to fail and how can we prevent ourselves from them.

\begin{itemize}
	\item Lack of business strategy and being too optimistic \textit{The business strategy of a company is the key point of it, so it should be completely defined and fixed in order to minimize failures while improvising it. We can get someone from the outside the project to help with business strategy and estimation, so we keep being realistic. If we still need some help with the business strategy, a council of experts should hired to receive some advice in this critical point.}.
	
	\item Tension inside the team, which later grows into continuous arguments, causing a bad and inefficient environment to work in. \textit{We shouldn't care too much from the project that we start behaving disrespectfully with the other members of the team when they make a mistake. We should take into account how we are behaving with our employees and colleagues in order to always treat them properly and boost a good working environment at the office. As well as we have to give a reprimand to someone who has made a mistake, we should also be comprehensive and patient with him/her and motivate that person to do a better work in the future.}

	\item Losing big investments from the clients. Giving our clients a wrong idea about the work we do in our company will make the clients refuse to give us the investment we need or hire our services in the future. Also, making a presentation with the client and giving him/her contradicting ideas will make a really bad impression about the organization of our company, resulting in the loss of that client. \textit{It is mandatory that the directives of the company have several meetings a month to coordinate efforts and agree in the direction they want to conduit the company. Contradiction of ideas in front of a client are completely inappropiate and out of place; those ideas should have been thoguht and fixed before meeting the client. We should coordinate efforts because our goal is impressing the client, not confusing him/her with contradicting ideas, which doesn't generate trust in our company.}
	
	\item Overcharge a single person with tasks he/she doesn't have any experience nor knowledge. For example, making a directive do the job of a legal representative. \textit{In order to avoid making bad decisions because of a lack of knowledge or experience in that field, we should hire the services of experts instead of trying to make everything on our own. Among these experts, there could be: lawyers, marketing sellers, accountants, etc. Expertise advice will make us more successful decisions for our company.}
	
	\item Lack of motivation of the team. If we want all the members of our team to give their best, we have to maintain them motivated. \textit{We might organize company activities for the whole team to reinforce relationships, which will result in a better working environment and a performance increase. Motivational talks given by elocuent directives will make the workers apprecciate their jobs and feel they are necessary for the company (of course, they are!). Moreover, making compliments and giving rewards to those people who do an exceptional work and always do their best, might be motivational boost.}

	\item Difficulty to set meetings given the different timetables of each person that has to attend that meeting, specially if they belong to different departments and ranks. \textit{It is necessary to find a moment of the day in which every attendant of the meeting is available, so meeting should not be improvised and instead, be timetabled in advance in order to avoid having any other issue during that moment.}

	\item Lack of senior executives who already have experience in other software development project. These people have done other projects that may have failed, so they could give the company an interesting and profitable point of view about critical points and how to avoid failures them in our project. \textit{Since we still don't have any previous failures, we don't really know what methods or procedures we have to apply to avoid them. We need to incorporate to our company employees who have already worked in order to receive advice from them or even let them participate in the organization and coordination of the team.}

	\item We haven't conducted a research on potential competitors' projects or applications already on the market before starting with our project. This might cause a lack of investment from the stakeholders because we don't really provide anything completely new to the market. \textit{It is very important to look around in the business field and watch out for competitors. One should not work in something that has been already developed and works well, if we can't give it a renovation or add it additional features. If we really want to succeed as a company, we should first analyze the market in order to detect needs or unsolved problems so we can innovate in our projects.}

	\item Inability to keep up with new technologies and techniques. If we can't learn in a short period of time how to use leading technology or methodologies, we will soon get behind the times, and clients will not trust in a newly-created but old-fashion company. \textit{Every enterprise, and even more those dedicated to technology, must keep on learning how technology is evolving and no adopt but adapt to the new systems.}

	\item Projects continuously failed because of bad timings and schedules or null assignment of tasks. \textit{We might have to set time limits, so concrete tasks should be done by those dates. This is a quite efficient way to know how things are going and if we are delayed in comparison with the development plan we had agreed to accomplish. To achieve this plan, tasks should be concretely defined and assign to specific members of the team, to see if everyone is working at the expected rythm.}

	\item Extremely fast growth of the company resulted in losing all control of it. We made good business and project plans at the start, but as soon as we accepted bigger projects and the company grew, roles in the company started difusing and it was a complete chaos. \textit{Roles in the company must be completely defined without any doubt, so that responsabilities for failures could be easily assigned and nobody would interfere in the working scope of another employee. Then, if the company is growing fast, then new plannings should be created at the same speed to maintain everything under control and keep our feet on the ground.}
\end{itemize}



\paragraph{Shifting the focus of the film to software project definition and start-up, what problems might arise and how could they be solved if the customer, developer and user are all the same person and if they are different people? What are the potential advantages of the customer, developer and user being the same person?}
\paragraph{}


The biggest advantage given by customer and developer being the same person is the impossibility of lack of communication. One should know exactly what he wants, so the requirements are clear from the very beginning. If not, then as soon as the person starts developing what he has in his mind, new requirements and ideas would start arising and he could be able to think about them earlier and change the design of the software if necessary. However, this requirement brainstorming might never end, enlarging the development phase indefinitely.

A similar thing happens if the developer and the user are the same person. 

Even easier communication would be if customer and user are also the same person. What the user needs is exactly what the customer would want to to be developed.

In summary, the usual communication problem would not exist at all if the developer, client and customer are the same person.



On the other hand, being customer, developer and user might cause several problems. As we mentioned before, the phase development might not finish in a decent period of time because while you are developing software, more and more useful requirements come to your mind and think they should be included in the software, so you don't finish developing.

Also, a problem of impartiality might arise sooner or later. The moment you assume so many responsabilities, you start losing objectivity to play those different roles at the same time.

Moreover, as you are the customer, you can freely set (or not set at all) a limit date for the delivery of the project, which it is likely to be continuously delayed as time goes by.

In conclusion, productivity and final results might get harmed if a single person assumes all those roles at the same time.



\paragraph{What qualities should or must a project leader have?}
\paragraph{}
The first of all, being \textbf{communicative}. Being able to demand the amount of work necessary to the workers, without make them slaves. A project leader should know the workers he works with to be able to demand work properly. The key to improve the coordination and organization of the work is the ability to assign every person the most appropiate task taking into account their abilities; and this is only achievable once you have taken your time to know your employees.

A project leader must know some technical stuff involved in the project (environmental knowledge) for a better comprehension of it and for a better understanding of problems that may arise. Another important quality is knowing how to solve problems and arguments that arise among the workers in order to maintain a good and comfortable working environment.

A good project leader must be able to make decision in a very short period of time; doubting instead of deciding doesn't generate trust in your employees nor in potential clients. This person must be able to deal with stress, because this job implies continuously working under the pressure of the clients and the date limits. 

Project leaders are sometimes the two faces of a coin: on the one hand, they must be very organized and strict in terms of project planning because that is the base for a project to success and not get delayed; on the other hand, they also have to be flexible to changes, because there might arise difficulties and plans don't always success.

Moreover, they are usually the public face of the team, so they should have a good level of sociability and negotation abilities in order to persuade potential clients and convince clients of the benefits of the developed software. A certain amount of creativity and innovation is quite convenient.

Project leaders stand out because of their teamwork and cooperation skills, which they use to motivate their teams and get the best from everyone designing a good project planning and task assignment that optimizes performance. Empathy is also desirable in a project leader.


\paragraph{What project management and leadership activities are performed in Startup.com?}
\paragraph{}

In Startup.com is performed a greedy aim to earn money and get clients lacking of any ethic nor moral principles. 

It is also performed a poor execution of plans, which at last caused delays in the launch of their website, letting competitors' websites surpass it. 

The leaders performed a very poor example to the workers because of the tension between the founders. If not even founders and chiefs can work together, we can't expect workers to work good enough.

Roles in the company were not totally defined, which caused that a member of the development team contradicted and argued with one of the founders of the company.

A good activity they did were the motivational talks to the whole team in order to get the best effort from every member of the team. Also, there was a talk to the team after one of the founders was fired, to explain the reasons why that happened. 

Another important leadership activity performed was the customer service and negotiation, as one of the founders had to deal with potential stakeholders in order to get an investment for their company. at the beginning of the documentary.



\paragraph{List the steps to be taken to define and start up a software project}
\paragraph{}

\begin{enumerate}
	\item Have a great idea.
	\item Check it is feasible and nobody else has done it.
	\subitem if that idea exists, think how to improve it and if it's necessary.
	\item Get a small great team.
	\item Do a exhaustive viability study to see if it is possible to start up the project and if it will be worthwhile.
	\item Get an office and some investors who can help you with first period of businesses where no money is earned.
	\item Start working, in developing (not just coding, analysis and design is developing too) and in commercial labors: someone will need to buy what you are doing so the businesses makes money. Even if the project is just some software someone has asked us to do, we need to do commercial labors talking to the hirer who hired us.
	\item Know how to deal with fast growing. Know how to let others do the job so chiefs don't have to be in charge of everything.
	\item Now you have made a great business, you can keep working and earning money.
\end{enumerate}

\end{document}
